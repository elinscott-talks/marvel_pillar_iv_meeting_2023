\documentclass[xcolor=table,aspectratio=169]{beamer}
\usetheme{Madrid}
\usepackage{adjustbox}
\usepackage{dcolumn}
\newcolumntype{d}[1]{D{.}{.}{#1}}
%\usetheme{metropolis}
\usepackage[style=verbose-note, sorting=none, sortcites=true, maxnames=1, giveninits=true, autocite=superscript, doi=false, url=false, isbn=false, backend=biber, citetracker=false, pagetracker=false, bibencoding=utf8, eprint=false]{biblatex}
% \usepackage[backend=bibtex,style=authoryear-comp,citestyle=authoryear-comp,firstinits=true,sorting=none,maxnames=1,doi=false,isbn=false,url=false,eprint=false]{biblatex}
\usepackage[T1]{fontenc}

\definecolor{twitter_blue}{HTML}{1da1f2}
\input{seaborn_colours.tex}

% Gobbling first names

\AtEveryCitekey{%
   \clearfield{shorttitle}%
   \clearfield{month}%
   \ifentrytype{article}{%
      \clearfield{title}%
   }{}
   }
\ExecuteBibliographyOptions[online]{eprint=true}

% "blindfootcite" is the equivalent of "footcite" except the number marker does not appear
\newcommand\blfootcite[1]{%
  \begingroup
  \renewcommand\thefootnote{}\footnote{\hspace{-4ex}\cite{#1}}%
  \addtocounter{footnote}{-1}%
  \endgroup
}
\renewcommand*{\multicitedelim}{\textcolor{seaborn_bg_grey_darker}{\addsemicolon}}
\setbeamerfont{footnote}{size=\scriptsize}
\renewcommand\footnoterule{\kern-3pt \color{seaborn_bg_grey_darker}\hrule width \textwidth height 0.4pt \color{black} \kern 2.6pt}

\DeclareSourcemap{
  \maps[datatype=bibtex,overwrite=False]{
   \map{
     \step[fieldsource=journal,
           match={Journal of Chemical Theory and Computation},
           replace={JCTC}]
     \step[fieldsource=journal,
           match={Reviews of Modern Physics},
           replace={Rev. Mod. Phys.}]
     \step[fieldsource=journal,
           match={Reports on Progress in Physics},
           replace={Rep. Prog. Phys.}]
     \step[fieldsource=journal,
           match={Physical Review Letters},
           replace={Phys. Rev. Lett.}]
     \step[fieldsource=journal,
           match={Physical Review},
           replace={Phys. Rev.}]
     \step[fieldsource=journal,
           match={B - Condensed Matter and Materials Physics},
           replace={B}]
     \step[fieldsource=journal,
           match={Journal of Chemical Physics},
           replace={J. Chem. Phys.}]
     \step[fieldsource=journal,
           match={Annual Review of Materials Research},
           replace={Annu. Rev. Mater. Res.}]
   }
  }
}

\renewbibmacro{in:}{}
\DeclareFieldFormat{pages}{\mkfirstpage{#1}}
\beamertemplatenavigationsymbolsempty
\bibliography{references.bib}
\setbeamertemplate{bibliography item}[text]
\renewbibmacro{in:}{}
\AtEveryBibitem{\clearfield{title}}
\AtEveryBibitem{\clearfield{month}}
\AtEveryBibitem{\clearfield{pages}}
\AtEveryBibitem{\clearfield{note}}
\DeclareNameAlias{default}{given-family}

% \renewcommand*{\bibfont}{\tiny}
\usepackage{amssymb}
\usepackage{epsfig}
\usepackage{psfrag}
\usepackage{wrapfig}
\usepackage{graphicx}
\usepackage{color}
\usepackage[table]{xcolor}
\usepackage{amsmath}
\usepackage{multimedia}
\usepackage{subcaption}
%\usepackage{style}
\usepackage{verbatim}
\usepackage{multicol}
\usepackage[table]{xcolor}
\usepackage{tabularx}
% Tikz
\usepackage{tikz}
\usetikzlibrary{positioning,shapes,arrows,backgrounds,fit,calc,external,trees}
% \tikzexternalize[prefix=tikzfigures/]
\tikzstyle{dummy} = []
\tikzstyle{line} = [draw, thick, -latex']
\tikzstyle{headless_line} = [draw, thick, -]
\tikzstyle{default}    = [rectangle, text centered, rounded corners, text=black, font=\sffamily\footnotesize, align=center]
\tikzstyle{default_text}    = [rectangle, text width=10cm, text=black,anchor=north west, font=\sffamily]
\tikzstyle{boxwhite} = [default, fill=white, rounded corners=0.1cm]
\tikzstyle{cp}    = [default, fill=seaborn_blue, text=white, text width=2.8cm, minimum height=0.5cm]
\tikzstyle{pw}    = [cp, fill=seaborn_green]
\tikzstyle{wannier90}    = [cp, fill=seaborn_cyan]
\tikzstyle{bespoke}    = [cp, fill=seaborn_magenta]
\tikzstyle{observable}    = [cp, fill=seaborn_red]
\tikzset{
  -|-/.style={
    to path={
      (\tikztostart) -| ($(\tikztostart)!#1!(\tikztotarget)$) |- (\tikztotarget)
      \tikztonodes
    }
  },
  -|-/.default=0.5,
  |-|/.style={
    to path={
      (\tikztostart) |- ($(\tikztostart)!#1!(\tikztotarget)$) -| (\tikztotarget)
      \tikztonodes
    }
  },
  |-|/.default=0.5,
}

\newlength{\myyshift}
\setlength{\myyshift}{0.05cm}

\usepackage{lipsum}
\usetikzlibrary{calc}
\newlength{\myfigscale}
\setlength{\myfigscale}{0.3cm}
\usepackage{smartdiagram}
\usesmartdiagramlibrary{additions}
\usepackage{multicol}
\usepackage{helvet}
% \usepackage{sansmath}
% \sansmath
\usepackage[normalem]{ulem} % for sout (strike out)
\usepackage{tcolorbox}
\tcbuselibrary{skins,hooks}
\tcbset{colframe=structure,fonttitle=\bfseries,beamer, clip upper, boxsep=0pt, sharp corners=all, no shadow, left skip=0pt, right skip=0pt, coltext=white}

% For electron orbital diagrams
\usepackage{tikzorbital}
% Changing defaults
\pgfkeys{tikzorbital/drawLevel/width = 0.666666}
\pgfkeys{tikzorbital/drawLevel/style = {line width = 1pt, color = black!80, line cap = round}}
\pgfkeys{tikzorbital/drawLevel/spinlength = 0.666666}
\pgfkeys{tikzorbital/drawLevel/spinstyle = {very thick, color = black!80, -stealth}}

\input{seaborn_colours.tex}

% For tikz diagrams with nodes appearing on each slide
\tikzset{
  invisible/.style={opacity=0},
  visible on/.style={alt={#1{}{invisible}}},
  alt/.code args={<#1>#2#3}{%
    \alt<#1>{\pgfkeysalso{#2}}{\pgfkeysalso{#3}} % \pgfkeysalso doesn't change the path
  },
}

\usepackage{array}
\usepackage{multirow}
% \newcolumntype{L}[1]{>{\raggedright\let\newline\\\arraybackslash\hspace{0pt}}m{#1}}
% \newcolumntype{C}[1]{>{\centering\let\newline\\\arraybackslash\hspace{0pt}}m{#1}}
% \newcolumntype{R}[1]{>{\raggedleft\let\newline\\\arraybackslash\hspace{0pt}}m{#1}}
\newcolumntype{L}{>{\raggedright\arraybackslash}X}
\newcolumntype{C}{>{\centering\arraybackslash}X}
\newcolumntype{R}{>{\raggedleft\arraybackslash}X}

% For checklist
%\usepackage{enumitem}
%\newlist{todolist}{itemize}{2}
%\setlist[todolist]{label=$\square$}
\usepackage{pifont}
\newcommand{\cmark}{\ding{51}}%
\newcommand{\xmark}{\ding{55}}%
\newcommand{\done}{\rlap{$\square$}{\raisebox{2pt}{\large\hspace{1pt}\cmark}}%
\hspace{-2.5pt}}
\newcommand{\wontfix}{\rlap{$\square$}{\large\hspace{1pt}\xmark}}

\newcommand{\bra}[1]{\langle #1|}
\newcommand{\braket}[2]{\langle #1|#2\rangle}
\newcommand{\braopket}[3]{\langle #1|#2|#3\rangle}
\newcommand{\ket}[1]{|#1\rangle}
\newcommand{\nline}{\nonumber \\}
\newcommand{\Trace}{\mathrm{Tr}}

\renewcommand{\ttdefault}{pcr} % enables bold fixed width font
\numberwithin{equation}{section}
% \usefonttheme{professionalfonts}
%\usefonttheme[stillsansseriflarge,stillsansserifsmall]{serif}
\usepackage{siunitx,booktabs}
% \AtBeginDocument{\sisetup{math-rm=\mathsf, text-rm=\sffamily}}
\AtBeginEnvironment{frame}{\setcounter{footnote}{0}}

\newlength{\myimscale}


% For code blocks in latex
% Taken from https://github.com/daveyarwood/gruvbox-pygments
% N.B.
%  - frame must have [fragile]
%  - use \begin{onlyenv} not \only
%  - after a lot of mucking around, I created gruvbox_plain as another style
%    that exclusively uses gruvbox's bg and fg with no syntax highlighting
%  - use [autogobble] to remove leading indentations

\usepackage{minted}
\usemintedstyle{gruvbox-dark}
\definecolor{gruvbox_dark_bg}{HTML}{282828}
\definecolor{gruvbox_fg}{HTML}{ebdbb2}
\definecolor{kgrey}{HTML}{2b2828}
\setminted[python]{bgcolor=gruvbox_dark_bg}
\setminted[json]{bgcolor=gruvbox_dark_bg}
\setminted[shell-session]{style=gruvbox_plain, bgcolor=gruvbox_dark_bg}

% \lstset{breaklines,breakatwhitespace,breakautoindent=false,showstringspaces=false}
% \lstset{keywordstyle=\color{purple}}
% \lstset{identifierstyle=\color{blue}}
% \lstset{basicstyle=\fontfamily{pcr}\fontsize{9pt}{9pt}\selectfont}
% %\lstset{numbers=left, numberstyle=\tiny, stepnumber=1, numbersep=5pt}
% \lstset{linewidth=4.9in,xleftmargin=10pt}

\setbeamercolor{frametitle}{bg=kgrey,fg=white}
\setbeamerfont{normal text}{family=helvet}
\setbeamerfont{local structure}{family=helvet}

\setbeamercolor*{author in head/foot}{bg=seaborn_blue}
\setbeamercolor*{logo in head/foot}{bg=seaborn_blue,fg=white}
\setbeamercolor*{title in head/foot}{bg=seaborn_blue,fg=white}
\setbeamercolor*{date in head/foot}{bg=seaborn_blue,fg=white}
\setbeamercolor{title}{bg=seaborn_blue}
\setbeamercolor{under headline}{bg=seaborn_red}
\setbeamercolor{footline}{bg=seaborn_blue}
\setbeamercolor{caption name}{fg=seaborn_blue}
\setbeamercolor{block title}{bg=kgrey,fg=white}
\setbeamercolor{block body}{bg=seaborn_bg_grey,fg=black}

% Footnote style and colour
% No line over footnote
\setbeamercolor{footnote}{fg=seaborn_bg_grey_darker}

\setbeamertemplate{enumerate items}[default]
\setbeamertemplate{blocks}[default]
\setbeamertemplate{itemize items}{\normalsize $\bullet$}
\setbeamercolor{description item}{fg=seaborn_blue}
\setbeamercolor{enumerate item}{fg=seaborn_blue}
\setbeamercolor{itemize item}{fg=seaborn_blue}
\setbeamercolor{itemize subitem}{fg=seaborn_blue}
\setbeamercolor{itemize subsubitem}{fg=seaborn_blue}
\setbeamercolor*{bibliography entry title}{fg=seaborn_bg_grey_darker}
\setbeamercolor*{bibliography entry author}{fg=seaborn_bg_grey_darker}
\setbeamercolor*{bibliography entry location}{fg=seaborn_bg_grey_darker}
\setbeamercolor*{bibliography entry note}{fg=seaborn_bg_grey_darker}
% and kill the abominable icon
\setbeamertemplate{bibliography item}[text]

\setbeamerfont*{title in head/foot}{size=\small}
\setbeamerfont*{date in head/foot}{size=\small}
\setbeamerfont*{institute}{size=\Large}

\setbeamertemplate{frametitle}
{
  \leavevmode%
  \vspace{-20pt}
  \begin{beamercolorbox}[wd=\paperwidth,ht=1cm]{frametitle}
   \hspace{0.115em}
   \vphantom{P/p} \bf \insertframetitle \vspace{0.2cm}
   \end{beamercolorbox}%
  %  \vskip-0.6cm%
  % \begin{beamercolorbox}[wd=\paperwidth,ht=0.5ex]{under headline}%
  %   \end{beamercolorbox}%
	
}

\newcommand{\insertframeinfo}{| \insertframenumber/\inserttotalframenumber}
\newcommand{\backupbegin}{
   \newcounter{finalframe}
   \setcounter{finalframe}{\value{framenumber}}
   \renewcommand{\insertframeinfo}{}
}
\newcommand{\backupend}{
   \setcounter{framenumber}{\value{finalframe}}
}


\setbeamertemplate{frametitle}
{
  \vspace{-1pt}
  \begin{beamercolorbox}[wd=\paperwidth,ht=0.8cm]{frametitle}
   \hspace{0.05em}
   \begin{minipage}{0.8\textwidth}
     \bf \insertframetitle

   \end{minipage}
   \hfill
   \begin{minipage}{0.15\textwidth}
   \begin{flushright}
   \scriptsize \textbf{Edward Linscott}
   
   \includegraphics[height=0.21cm]{logos/white_cropped.eps}
   \textbf{\insertframeinfo}
   \end{flushright}
   \end{minipage}
   \vspace{0.125cm}
  \end{beamercolorbox}%
}

\setbeamertemplate{title page}
{
  \leavevmode%
  \vbox{%
  \vspace{-1.6ex}%
  \noindent\begin{tcolorbox}[enhanced,watermark graphics=photos/EPFL-Leman-vue-aerienne-1536x864.jpg, width=\paperwidth, height=0.57\paperwidth, watermark zoom=1.25, grow to left by=0.035\paperwidth, frame hidden]

  \vspace{1.5ex}
  \begin{minipage}{\textwidth}
   \begin{flushright}
   \includegraphics[height=0.05\textheight]{figures/logo_marvel_color_transparent.png}
   \hspace{0.1ex}
   \includegraphics[height=0.05\textheight]{logos/SNF_logo_standard_web_color_pos_e.png}
   % \hspace{0.01\textheight}
   % \includegraphics[height=0.05\textheight]{logos/black_cropped.eps}
   \hspace{0.1cm}\hbox{}
  \end{flushright}

  \vspace{2.5em}
  \begin{center} 
  \huge
  \textbf{Koopmans functionals}

  \normalsize
  \textbf{accurately and efficiently predicting spectral properties \\ with a functional formulation}
  \end{center}
  \end{minipage}
  \end{tcolorbox}

  \vspace{-2em}
  \begin{tcolorbox}[width=\paperwidth, enhanced, colback=kgrey, grow to left by=0.035\paperwidth,]
  \begin{center}
  \footnotesize \bf \insertauthor\quad | \quad\insertshortinstitute\quad | \quad CECAM flagship workshop, Lausanne\quad|\quad \insertdate    
  \end{center}
  %  \end{flushright}
  \end{tcolorbox}
  }


	
}
%\setbeamerfont{frametitle}{series=\bfseries}
\setbeamertemplate{footline}
{
}

% Title slide %%%%%%%%%%%%%%%%%%%%%%%%%%%%%%%%%%%%%%%%%%%%%%%%%%%%%%%%%%%%%%%%%%%%%%%%%%%%%%%%%%%
% \title[Beamer Intro]{\noindent Strongly correlated biological systems\\\Large DFT and beyond}
\title{\noindent\large{an open-source package for accurately predicting spectral properties}}
\author{Edward Linscott}
\institute{EPFL}
\date{25 Aug 2023}
\begin{document}

\frame{\titlepage}

% \begin{frame}{Take home messages}
% 
%    Koopmans functionals are a class of functionals that aim to reproduce spectral properties and total energies on the same footing
% 
%    \vspace{1em}
%    (They share many parallels --- and have subtle differences --- with other methods presented this week)
% 
%    \vspace{1em}
%    As a result they give band structures and orbital energies comparable to state-of-the-art GW
% 
%    \vspace{1em}
%    We have released \texttt{koopmans}, a package that contains everything necessary to run calculations using Koopmans functionals without expert knowledge
% 
%    % Koopmans functionals are a class of functionals that aim to reproduce spectral properties (charged excitations) and total energies on the same footing by enforcing a generalized piecewise linearity condition.
% 
%    % Koopmans functionals give band structures and orbital energies as accurate as state-of-the-art GW, at a fraction of the computational cost
% 
% \end{frame}

% \begin{frame}{Failures of DFT}
%    \begin{itemize}
%       \item band gap is almost universally too small
%       \item problems with ``strongly correlated" systems (e.g. TMOs)
%       \item problems with vdW interactions
%       \item eigenvalues are formally meaningless
%       \item self-interaction error
%       \item static correlation error
%    \end{itemize}
%    % \onslide<3->{
%    % \begin{equation*}
%    %    V^x_i\psi_{n_i}(\mathbf{r}) = -\frac{1}{2}\sum_{i\neq j} \psi_{n_j}(\mathbf{r'})\int d\mathbf{r}' \frac{\psi^*_{n_j}(\mathbf{r}')\psi_{n_i}(\mathbf{r}')}{|\mathbf{r} - \mathbf{r}'|}
%    % \end{equation*}
%    % }
% \end{frame}

% \begin{frame}{Self-interaction error}
%    \nocite{Dabo2010,Borghi2014,Nguyen2018,Colonna2018,Colonna2019,DeGennaro2022,Colonna2022}
%    % Koopman's theorem\blfootcite{Koopmans1934a}: changes in energy upon removal of electrons from unrelaxed HF orbitals: $\epsilon_{i\sigma} = - \Delta E_{i\sigma}$
%    % % at least for unrelaxed Hartree Fock
% 
%    \begin{overlayarea}{\textwidth}{0.7\textheight}
%       \begin{center}
%          \begin{onlyenv}<1>
%             \includegraphics[height=0.7\textheight]{figures/curvature_plot/fig_en_curve_with_all.pdf}
%          \end{onlyenv}
% 
% 
%          \begin{onlyenv}<2->
%             \includegraphics[height=0.7\textheight]{figures/curvature_plot/fig_en_curve_sl_annotated_zoom.pdf}
%          \end{onlyenv}
%       \end{center}
% 
%    \end{overlayarea}
% 
%    \blfootcite{Cohen2008,Li2017}
% 
% \end{frame}

\begin{frame}{Koopmans functionals: theory}
   \hbox{
      \begin{minipage}{0.5\textwidth}
         Goal: spectral properties (charged excitations) with a functional theory

         \onslide<2->{
         \vspace{1ex}
         For the exact Green's function, we have poles that correspond to total energy differences
         \begin{equation*}
            \varepsilon_i =
            \begin{cases}
            E(N) - E_i(N-1) & i \in \text{occ} \\
            E_i(N+1) - E(N) & i \in \text{emp}
            \end{cases}
         \end{equation*}
         }

         \onslide<3->{
         \vspace{1ex}
         For DFT, this condition is \emph{not} satisfied in general
         }
         %
      \end{minipage}

      \begin{minipage}{0.45\textwidth}
         \centering
         \only<1-2>{
            \includegraphics[height=0.7\textheight]{figures/photoemission_costantini.png}
         }
         \only<3>{
            \includegraphics[width=\columnwidth]{figures/curvature_plot/fig_en_curve_gradients_zoom.pdf}
         }
      \end{minipage}
   }
   \onslide<1-2>{
      \blfootcite{Costantini2020}
   }
\end{frame}

\begin{frame}{Koopmans functionals: theory}
   \hbox{
      \begin{minipage}{0.5\textwidth}
            Core idea: for every orbital $i$ their energy
            \begin{equation*}
               \varepsilon^\mathsf{Koopmans}_i = \braopket{\varphi_i}{H}{\varphi_i} = \partial E_\mathsf{Koopmans}/\partial f_i
            \end{equation*}
            ought to be...
            \begin{itemize}
               \item independent of its own occupation $f_i$
               \item equal to the corresponding total energy difference $E_i(N-1) - E(N)$
            \end{itemize}
         %
      \end{minipage}

      \begin{minipage}{0.45\textwidth}
         \centering
         \only<1>{
            \includegraphics[width=\columnwidth]{figures/curvature_plot/fig_en_curve_gradients_zoom.pdf}
         }
         % \only<2>{
         %    \includegraphics[height=0.6\textheight]{figures/fig_gw100_dscf_cf_ks.png}
         % }
      \end{minipage}
   }
\end{frame}

\begin{frame}{Koopmans functionals: theory}

   \vspace{-1.5ex}
   \begin{overlayarea}{\textwidth}{0.3\paperheight}
      \only<1-7>{
   \begin{align*}
      E_\mathsf{Koopmans}[\rho,\only<8->{\textcolor{red}}{\{f_i\}}, \only<6>{\textcolor{red}}{\{\alpha_i\}}]
      = \only<2>{\textcolor{red}}{E_{DFT}[\rho]}
      + \sum_i
      \only<6>{\textcolor{red}}
      {\alpha_i}
      \Biggl(
      -
      \underbrace{
         \only<3>{\textcolor{red}}{\int^{f_i}_{0} \varepsilon_i(f) df}
      }_{
         \substack{\text{removes}  \\ \text{curvature}}
      }
      +
      \underbrace{
         \only<4>{\textcolor{red}}{f_i \only<7>{\textcolor{red}}{\eta_i}}
      }_{
         \substack{\text{restores} \\ \text{linearity}}
      }
      \Biggr)
   \end{align*}
      }
      \only<8->{

   \begin{align*}
      E_\mathsf{KI}[\rho,\only<9->{\textcolor{red}}{\{\rho_i\}}, \only<6>{\textcolor{red}}{\{\alpha_i\}}]
      = \only<2>{\textcolor{red}}{E_{DFT}[\rho]}
      + \sum_i
      \only<6>{\textcolor{red}}
      {\alpha_i}
      \biggl(&
      E_{\rm Hxc} [\rho-\only<9>{\textcolor{red}}{\rho_i}] -E_{\rm Hxc}[\rho] \nonumber \\
      & \qquad + f_i \left( E_{\rm Hxc}[\rho-\only<9>{\textcolor{red}}{\rho_i}+\only<9>{\textcolor{red}}{n_i}] -E_{\rm Hxc}[\rho-\only<9>{\textcolor{red}}{\rho_i}] \right)
      \biggr)
   \end{align*}
      }
\end{overlayarea}

   \begin{overlayarea}{\textwidth}{0.5\paperheight}
      \centering
      \only<1-4>{
         \vspace{-1em}
      }
      \only<1>{
         \includegraphics[height=0.5\textheight]{figures/curvature_plot/fig_en_curve_koopmans_step0.pdf}
      }

      \only<2>{
         \includegraphics[height=0.5\textheight]{figures/curvature_plot/fig_en_curve_koopmans_step1.pdf}
      }

      \only<3>{
         \includegraphics[height=0.5\textheight]{figures/curvature_plot/fig_en_curve_koopmans_step2.pdf}
      }

      \only<4>{
         \includegraphics[height=0.5\textheight]{figures/curvature_plot/fig_en_curve_koopmans_step3.pdf}
      }

      \only<5->{
         \raggedright
         Features:
         \begin{itemize}
            \item<6-> screening (calculated ab initio)
            \item<7-> different variants: KI (leaves total energy unchanged), KIPZ (exact for 1-electron systems), pKIPZ
            \item<9-> orbital-density dependence
         \end{itemize}
      }
      % \only<8->{
      %    \begin{equation*}
      %       v^\mathrm{KI}_{\textcolor{red}{i}}/\alpha_i = - E_{\mathrm{H}}\left[\textcolor{red}{n_{i}}\right]
      %       + E_{\mathrm{xc}}\left[\rho\right]
      %       - E_{\mathrm{xc}}\left[\rho-\textcolor{red}{n_{i}}\right]
      %       - \int d\mathbf{r'}
      %       v_\mathrm{xc}(\mathbf{r}', [\rho])
      %       \textcolor{red}{n_{i}}(\mathbf{r}')
      %    \end{equation*}
      % }
      % \only<6-7>{
      %    \begin{equation*}
      %       \frac{d E}{d f_i}
      %       \approx
      %       \alpha_i \frac{\partial E}{\partial f_i}
      %       \onslide<7->{
      %          \Rightarrow \varepsilon_i^\mathsf{Koopmans} = \frac{\partial E_\mathsf{Koopmans}}{\partial f_i}  \approx E_i(N-1) - E(N)}
      %    \end{equation*}
      % }

   \end{overlayarea}
   \blfootcite{Dabo2010,Borghi2014,Colonna2019}
\end{frame}


\begin{frame}{Koopmans functionals: theory}
   Consequences of ODD:
   \begin{itemize}[<+->]
      \item a natural generalisation in the direction of spectral functional theory\footcite{Ferretti2014}
      \item variational (localised, minimising) vs canonical (delocalised, diagonalising) orbitals
            \begin{figure}[t]
               \centering
               \begin{subfigure}{0.3\textwidth}
                  \includegraphics[height=\columnwidth,angle=90]{figures/fig_nguyen_variational_orbital.png}
                  \caption{variational}
               \end{subfigure}
               \hspace{0.1\textwidth}
               \begin{subfigure}{0.3\textwidth}
                  \includegraphics[height=\columnwidth,angle=90]{figures/fig_nguyen_canonical_orbital.png}
                  \caption{canonical}
               \end{subfigure}
            \end{figure}
      \item Practically we can often use MLWFs
      \item localised variational orbitals naturally allow us to treat bulk systems
      % \item ODD functional means that we know $\hat H \ket{\varphi_i}$ for variational orbitals $\{\ket{\varphi_i}\}$ but we don't know $\hat H$ in general
   \end{itemize}
   \blfootcite{Nguyen2018}
\end{frame}

\begin{frame}{Koopmans functionals: theory}

   \begin{center}
      \includegraphics[width=0.8\textwidth]{figures/fig_nguyen_scaling.png}
   \end{center}
   \blfootcite{Nguyen2018}


\end{frame}

% \begin{frame}{Koopmans functionals: comparing}
%    \small
%    \renewcommand{\arraystretch}{1.5}
%    \rowcolors{1}{seaborn_bg_grey}{seaborn_bg_grey_half}
%    \begin{tabularx}{\columnwidth}{L L L}
%                                                 & \textbf{DFT+\emph{U}}                                                       & \textbf{Koopmans}                                                                                                                                                   \\
%       \hline
%       designed to correct SIE, as defined by... & erroneous global curvature in total energies                                & dependence of $\varepsilon_i$ on $f_i \ \forall i$ \leavevmode\onslide<4->{\textcolor{red}{(canonical orbitals)}}                                                   \\
%       by construction...                        & corrects local curvature in total energies                                  & removes dependence of $\varepsilon_i$ on $f_i$ and guarantees $\varepsilon_i = E_i(N\pm 1) - E(N)$ \leavevmode\onslide<4->{\textcolor{red}{(variational orbitals)}} \\
%       correction applied to...                  & selected subspaces only (e.g. \emph{3d} orbitals)                           & the entire system                                                                                                                                                   \\
%       orbitals defined by...                    & Hubbard projectors (atom-centred, frozen, incomplete)                       & \leavevmode\onslide<2->{variational (minimising) orbitals}                                                                                                          \\
%       corrective parameters are...              & $\{U^I\}$, defined with respect to charge-neutral excitations (if using LR) & \leavevmode\onslide<3->{$\{\alpha_i\}$, defined with respect to charged excitations}                                                                                \\
%    \end{tabularx}
% \end{frame}

\begin{frame}{Koopmans functionals: theory}
   Resonance with other efforts:
   \begin{itemize}
      \item Wannier transition-state method of Anisimov and Kozhevnikov \cite{Anisimov2005}
      \item Optimally tuned hybrid functionals of Kronik, Pasquarello, and others \cite{Kronik2012,Wing2021}
      \item Ensemble DFT of Kronik and co-workers \cite{Kraisler2013}
      \item Koopmans-Wannier of Wang and co-workers \cite{Ma2016}
      \item Dielectric-dependent hybrid functionals of Galli and co-workers \cite{Skone2016a}
      \item LOSC functionals of Yang and co-workers \cite{Li2018}
   \end{itemize}
\end{frame}

\begin{frame}{Koopmans functionals: results for molecules}
   \small
   Ionisation potentials $ = E(N-1) - E(N) \stackrel{?}{=} -\varepsilon_{HO}$ of 100 molecules (the GW100 set) cf. CCSD(T)
   \begin{center}
      \includegraphics[height=0.2\textwidth]{figures/colonna_2019_gw100_ip}
      % \onslide<2->{\includegraphics[height=0.23\textwidth]{figures/colonna_2019_gw100_deeper}}
   \end{center}

   \vspace{-3ex}
   Ultraviolet photoemission spectra
   \begin{center}
      \begin{tikzpicture}
         \node [inner sep=0pt](fig) at (0,0) {\includegraphics[height=0.35\textheight]{figures/fig_nguyen_prl_spectra.png}};
         \draw [very thick, color=seaborn_red] (-5.35,-0.07) rectangle (5.4,1.6);
      \end{tikzpicture}
   \end{center}
   \vspace{-2ex}

   \blfootcite{Colonna2018,Nguyen2015}
\end{frame}

% \begin{frame}{Koopmans functionals: results for molecules}
%    Electron affinities $ = E(N) - E(N+1) \stackrel{?}{=} -\varepsilon_{LU}$ of molecules cf. CCSD(T)/exp
%    \vspace{2ex}
% 
%    \small
%    \begin{center}
%       For 15 of the GW100 molecules with bound LUMOs
% 
%       \includegraphics[height=0.5\textheight]{figures/fig_gw100_ea_mae_mse.pdf}
% 
%       \textcolor{seaborn_bg_grey_darker}{\footnotesize Linscott et al. (in prep)}
%    \end{center}
% \end{frame}

\begin{frame}{Koopmans functionals: results for solids}
   \begin{minipage}[c]{0.35\textwidth}
      \includegraphics[width=\textwidth]{figures/fig_nguyen_prx_bandgaps.png}
   \end{minipage}
   \hspace{1em}
   \begin{minipage}[c]{0.6\textwidth}

      \footnotesize
      Mean absolute error (eV) across prototypical semiconductors and insulators

      \vspace{1ex}
      \begin{tabular}{c S[table-format = 2.2] S[table-format = 2.2] >{\color{seaborn_red}\bfseries}S[table-format = 2.2] >{\color{seaborn_red}\bfseries}S[table-format = 2.2] S[table-format = 2.2]}
                          & {PBE} & {G\textsubscript{0}W\textsubscript{0}} & {KI} & {KIPZ} & {QSG$\tilde{\mathrm{W}}$} \\
         \midrule
         \midrule
         $E_\mathrm{gap}$ & 2.54  & 0.56                                   & 0.27 & 0.22   & 0.18                      \\
         %                                  & {MAPE (\%)} & 48.28 & 12.10      & 7.0           \\
         \midrule
         IP               & 1.09  & 0.39                                   & 0.19 & 0.21   & 0.49                      \\
         %                                  & {MAPE (\%)} & 15.58 & 5.71                                   & 2.99 & 3.14   & 7.41
      \end{tabular}
   \end{minipage}

   \blfootcite{Nguyen2018}
\end{frame}

\begin{frame}{Koopmans functionals: results for solids}
   
\begin{table}[t]
   \centering
   \footnotesize
   \begin{tabular}{r@{ $\rightarrow$ } l *{3}{d{2.2}} >{\color{seaborn_red}}S[table-format = 2.2] >{\color{seaborn_red}}S[table-format = 2.2] d{2.2} @{$\pm$} d{1.2}}
      \hline
      \hline
      \multicolumn{2}{c}{ }
                                & \multicolumn{1}{c}{PBE}
                                & \multicolumn{1}{c}{G\textsubscript{0}W\textsubscript{0}\footnote{\cite{Shishkin2007} for $E_g$ and \cite{Hybertsen1986} for the transitions;}}
                                & \multicolumn{1}{c}{scG$\tilde{\mathrm{W}}$\footcite{Shishkin2007a}}
                                & \multicolumn{1}{c}{
                                 \textcolor{seaborn_red}{\bfseries KI@[PBE,MLWFs]}}
                                & \multicolumn{1}{c}{
                                 \textcolor{seaborn_red}{\bfseries KIPZ@PBE}}
                                & \multicolumn{2}{c}{exp\footcite{Madelung2004}}                                                                                                                                                                   \\
      \hline
      \multicolumn{2}{c}{$E_g$} &
      0.49 &  1.06 & 1.14 &  1.16 &   1.15 & \multicolumn{2}{c}{1.17}\\
      $\Gamma_{1v}$ & $\Gamma_{25'v}$ & 11.97 & 12.04 &      & 11.97 & 12.09 & 12.5 &  0.6\\
      $X_{1v}$ & $\Gamma_{25'v}$ &  7.82 &       &      &  7.82       &       & \multicolumn{2}{c}{7.75}\\
      $X_{4v}$ & $\Gamma_{25'v}$ &  2.85 &  2.99 &      &  2.85 & 2.86 & \multicolumn{2}{c}{2.90}\\
      $L_{2'v}$ & $\Gamma_{25'v}$ &  9.63 &  9.79 &      &  9.63 &  9.74 &  9.3 &  0.4\\
      $L_{1v}$ & $\Gamma_{25'v}$ &  6.98 &  7.18 &      &  6.98 &   7.04 &  6.8 &  0.2\\
      $L_{3'v}$ & $\Gamma_{25'v}$ &  1.19 &  1.27 &      &  1.19 &       &  1.2 &  0.2\\
      $\Gamma_{25'v}$ &  $\Gamma_{15c}$ &  2.48 &  3.29 &      &  3.17  &  3.20 & 3.35 & 0.01\\
      $\Gamma_{25'v}$ &  $\Gamma_{2'c}$ &  3.28 &  4.02 &      &  3.95  &  3.95 & 4.15 & 0.05\\
      $\Gamma_{25'v}$ &        $X_{1c}$ &  0.62 &  1.38 &      &  1.28  &  1.31 & \multicolumn{2}{c}{1.13} \\
      $\Gamma_{25'v}$ &        $L_{1c}$ &  1.45 &  2.21 &      &  2.12  &  2.13 & 2.04 & 0.06\\
      $\Gamma_{25'v}$ &        $L_{3c}$ &  3.24 &  4.18 &      &  3.91  &  3.94 &  3.9 &  0.1\\
      \hline
      \multicolumn{2}{c}{MSE} & 0.35 &  0.02 &      &  0.01 &   0.03\\
      \multicolumn{2}{c}{MAE} & 0.44 &  0.21 &      &  0.14 &   0.17\\
      \hline
      \hline
   \end{tabular}

   % \textsuperscript{\emph{a}} this work;
   % \textsuperscript{\emph{b}} Ref.~\citenum{Shishkin2007} for $E_g$ and Ref.~\citenum{Hybertsen1986} for the transitions;
   % \textsuperscript{\emph{c}} Ref.~\citenum{Shishkin2007a};
   % \textsuperscript{\emph{d}} Ref.~\citenum{DeGennaro2022};
   % \textsuperscript{\emph{e}} Ref.~\citenum{Madelung2004}
\end{table}
\end{frame}

\begin{frame}{Koopmans functionals: results for solids}
   \begin{figure}[t]
      \centering
      \begin{subfigure}{0.3\textwidth}
         \includegraphics[width=\columnwidth]{figures/ZnO_lda.png}
      \end{subfigure}
      \begin{subfigure}{0.3\textwidth}
         \includegraphics[width=\columnwidth]{figures/ZnO_hse.png}
      \end{subfigure}
      \begin{subfigure}{0.3\textwidth}
         \includegraphics[width=\columnwidth]{figures/ZnO_ki.png}
      \end{subfigure}
      \begin{subfigure}{\textwidth} %<-- changed width
         \centering
         %    \renewcommand\tabularxcolumn[1]{m{#1}}% <-- added
         %    \renewcommand\arraystretch{1.3}
         %    \setlength\tabcolsep{2pt}% <-- added
         \begin{tabular}{c S[table-format = 2.2] S[table-format = 2.2] S[table-format = 2.2] S[table-format = 2.2] >{\color{seaborn_red}\bfseries}S[table-format = 2.2] S[table-format = 2.2]}
            ZnO                                  & {LDA} & {HSE} & {GW$_0$} & {scG$\tilde{\rm W}$} & {KI} & {exp}       \\
            \hline
            $E_\mathrm{gap}$ (eV)                & 0.79  & 2.79  & 3.0      & 3.2                  & 3.62 & 3.60        \\
            $\langle \varepsilon_d \rangle$ (eV) & -5.1  & -6.1  & -6.4     & -6.7                 & -6.9 & {-7.5/-8.0} \\
         \end{tabular}
         %        \caption{table}
      \end{subfigure}
      % \caption{Band structure of ZnO calculated at different level of theory:
      %    LDA (left panel), HSE (middle panel) and KI (right panel). Shaded areas
      %    highlight valence (light blue) and conduction (light red) manifolds. The
      %    experimental values for the band gap and for the energy position of
      %    Zn $d$-states are represented by the dashed green line and by the dashed
      %    red line, respectively.
      %    Table: Band gap and position of Zn $d$ states with respect to the top of the valence band at different level of theory compared to experimental and GW results from Ref.~\onlinecite{shishkin_accurate_2007}.}
   \end{figure}
   \blfootcite{Colonna2022}
\end{frame}


\begin{frame}{Koopmans functionals: results for toy systems}
   For Hooke's atom (two electrons in a harmonic confining potential with Coulombic repulsion)

   \begin{figure}[t]
      \begin{subfigure}{0.4\textwidth}
         \includegraphics[width=\columnwidth]{figures/schubert_vxc.jpeg}
      \end{subfigure}
      \begin{subfigure}{0.4\textwidth}
         \onslide<2->{
         \includegraphics[width=\columnwidth]{figures/schubert_vxc_integrated.jpeg}
         }
      \end{subfigure}
   \end{figure}

   \blfootcite{Schubert2023}
\end{frame}

\begin{frame}{Koopmans functionals: caveats}

   \begin{itemize}[<+->]
      \item will \emph{not} solve H$_2^+$!
      \item restricted to systems with a non-zero band gap
      \item empty state localization in the bulk limit
      \item can potentially break the crystal point group symmetry\blfootcite{Su2020}
   \end{itemize}
\end{frame}


\begin{frame}{Accelerating improvements}
   The general workflow:
   \begin{itemize}
      \item define/initialize a set of variational orbitals
      \item calculate the screening parameters $\{\alpha_i\}$
      \item construct and diagonalize the Hamiltonian
   \end{itemize}

   Recent advances make some of these steps a lot easier...
\end{frame}

\begin{frame}{Accelerating improvements: easier Wannierization}

   \begin{figure}[t]
      \begin{subfigure}{0.225\textwidth}
         \includegraphics[height=1.5\columnwidth]{figures/proj_disentanglement_fig1b.png}
         \vspace{-0.01\paperheight}
      \end{subfigure}
      \begin{subfigure}{0.225\textwidth}
         \onslide<2->{
         \includegraphics[height=1.5\columnwidth]{figures/proj_disentanglement_fig1a.png}
         }
      \end{subfigure}
      \hspace{0.025\textwidth}
      % \begin{subfigure}{0.225\textwidth}
      %    \onslide<3->{
      %    \includegraphics[height=1.5\columnwidth]{figures/proj_disentanglement_fig1d.png}
      %    }
      % \end{subfigure}
      \begin{subfigure}{0.225\textwidth}
         \onslide<4->{
         \includegraphics[height=1.5\columnwidth]{figures/proj_disentanglement_fig1f.png}
         }
      \end{subfigure}
   \end{figure}

   \onslide<5->{Demonstrated on $>$20,000 materials $\rightarrow$ black-box Wannierization!}

   \blfootcite{Qiao2023}

\end{frame}

\begin{frame}{Accelerating improvements: screening via DFPT}
   Original formulation requires explicit charged defect calculations in a supercell

   \begin{equation*}
   \alpha^{n+1}_i =
   \alpha^n_i \frac{\Delta E^\text{Koopmans}_i - \lambda_{ii}(0, 1)}{\lambda_{ii}(\alpha^n_i, 1) - \lambda_{ii}(0, 1)}; \qquad  \Delta E^\text{Koopmans}_i = E^\text{Koopmans}(N) - E^\text{Koopmans}_i(N - 1)
   \end{equation*}

   \onslide<2->{
   Now reformulated in terms of DFPT\footcite{Colonna2019}...

   \begin{equation*}
   \alpha_{i} = 1 + \frac{\langle v^{i}_{\rm pert} \vert \Delta^{i} n \rangle}{\langle n_{i} \vert v^{i}_{\rm pert} \rangle}.
   \end{equation*}
   }
   
   \onslide<3->{
   ... in reciprocal space\footcite{Colonna2022}
   \begin{equation*}
      \alpha_{\mathbf{0}i} =  1 + \frac{\sum_{\mathbf{q}} \langle v^{\mathbf{0}i}_{\rm pert,\mathbf{q}} \vert \Delta^{\mathbf{0}i}_{\mathbf{q}}n \rangle} {\sum_{\mathbf{q}} \langle n^{\mathbf{0}i}_{\mathbf{q}} \vert v^{\mathbf{0}i}_{\rm pert,\mathbf{q}} \rangle}.
   \end{equation*}
   }

\end{frame}

\begin{frame}{Accelerating improvements: screening via ML}
   \begin{center}

      \includegraphics[height=0.7\paperheight]{figures/CsSnI3_calc_vs_pred_Edward.png}
      \includegraphics[height=0.7\paperheight]{figures/convergence_analysis_Edward.png}

      loss of accuracy of the band gap of $\sim$ 0.02 eV

      (cf. when calculating screening parameters \emph{ab initio})

      speedup of 70$\times$
   \end{center}

   \blfootcite{Schubert2022}

\end{frame}

\begin{frame}{Accelerating improvements: automated workflows}
   \small We have complicated workflows, with either...

   \vspace{1ex}

   \onslide<2->{
      (a) finite difference calculations using a supercell

      \vspace{-2ex}
      \adjustbox{width=\textwidth}{\input{supercell_workflow.tex}\end{tikzpicture}}
   }

   \vspace{-1.5ex}
   \onslide<3->{
      (b) DFPT using a primitive cell

      \vspace{-2ex}
      \adjustbox{width=0.655\textwidth}{\input{primitive_workflow.tex}}
   }

   \blfootcite{Linscott2023}
   % \onslide<6>{
   %    \vspace{-0.375\paperheight}
   %    \begin{flushright}
   %       \begin{tcolorbox}[enhanced jigsaw, width=4cm, opacityback=0, colframe=seaborn_red, coltext=seaborn_red, left=3pt, bottom=3pt, top=3pt, right=3pt, tikz={rotate=30,transform shape}, boxrule=1.5mm]
   %          \begin{center}
   %             \includegraphics[height=1cm]{./figures/qe_logo_high_res_cropped.jpg}
   %             \bf \huge\ \raisebox{0.3cm}{+}\,
   %             \includegraphics[height=1cm]{./figures/python_logo.png}

   %             \bf \large OUT NOW!
   %          \end{center}
   %       \end{tcolorbox}
   %    \end{flushright}
   % }

\end{frame}


\begin{frame}{Accelerating improvements: automated workflows}

   % \texttt{kcw.x} (DFPT implementation) is distributed in Quantum ESPRESSO v7.1 onwards

   % \vspace{4ex}

   Complicated workflows mean that...
   \begin{itemize}
      \item lots of different codes that need to handshake
      \item lots of scope for human error
      \item reproducibility becomes difficult
      \item expert knowledge required
   \end{itemize}

   Our solution...

\end{frame}

\begin{frame}{}
   \begin{center}
      \includegraphics[width=0.6\textwidth]{figures/koopmans_grey_on_transparent.png}
   \end{center}

   \vspace{-2ex}

   \begin{columns}
      \begin{column}{0.55\textwidth}
         \begin{itemize}
            \item v1.0 released earlier this year\footnotemark[1]
            \item implementations of Koopmans functionals within Quantum ESPRESSO
            \item automated workflows
                  \begin{itemize}
                     \item start-to-finish Koopmans calculations
                     \item Wannierisation
                     \item dielectric tensor
                     \item convergence tests
                     \item ...
                  \end{itemize}
            \item built on top of ASE\footnotemark[2]
            \item does not require expert knowledge
         \end{itemize}
      \end{column}

      \begin{column}{0.4\textwidth}
         \centering
         \url{koopmans-functionals.org}
         \includegraphics[width=\columnwidth]{figures/website_cropped.png}
      \end{column}
   \end{columns}
   \footnotetext[1]{\cite{Linscott2023}}
   \footnotetext[2]{\cite{Larsen2017}}
\end{frame}

% \begin{frame}{Neutral excitations with Koopmans}
%    \centering
%    \includegraphics[width=0.5\textwidth]{figures/elliott_workflow.jpeg}
%    \blfootcite{Elliott2019}
% \end{frame}
% 
% \begin{frame}{Neutral excitations with Koopmans}
%    \begin{columns}
%       \begin{column}{0.4\textwidth}
%          \includegraphics[width=\textwidth]{figures/elliott_scatter2.jpeg}
%       \end{column}
%       \begin{column}{0.6\textwidth}
%          \footnotesize
%          \begin{tabular}{c >{\color{seaborn_red}\bfseries}S[table-format = 2.2] S[table-format = 2.2] S[table-format = 2.2] S[table-format = 2.2] S[table-format = 2.2]}
%             \multirow{2}{*}{Thiel's set} & {KI-BSE} & \multicolumn{2}{c}{G$_0$W$_0$-BSE} & \multicolumn{2}{c}{TDDFT}                   \\
%                                          & {PBE}    & {PBE}                              & {B3LYP}                   & {PBE} & {B3LYP} \\
%             \hline
%             MAE (eV)                     & 0.54     & 0.83                               & 0.46                      & 0.55  & 0.27
%          \end{tabular}
% 
%          \vspace{2ex}
%          \onslide<2->{
%             Based on RPA $\rightarrow$ room for improvement with finite-field approaches
%          }
% 
%       \end{column}
%    \end{columns}
%    \centering
% 
%    \vspace{1ex}
%    \blfootcite{Elliott2019}
%    \blfootcite{Nguyen2019}
% 
% \end{frame}

\begin{frame}{Take home messages}

   \includegraphics[height=0.2\paperheight]{figures/colonna_2019_gw100_ip.jpeg}
   \hfill
   \includegraphics[height=0.2\paperheight]{figures/fig_nguyen_prx_bandgaps.png}
   \hfill
   \adjustbox{height=0.2\paperheight}{\input{supercell_workflow.tex}\end{tikzpicture}}

   \begin{itemize}
      \item Koopmans functionals are a class of functionals that treat spectral properties on the same footing as total energy differences (via GPWL)
      \item they can give orbital energies and band structures with comparable accuracy to state-of-the-art GW
      \item the release of \texttt{koopmans} means you don't need expert knowledge to run Koopmans functional calculations
   \end{itemize}

\end{frame}

\begin{frame}{Take home messages}
   \includegraphics[width=\textwidth]{figures/jctc.png}
   
\end{frame}

\begin{frame}{Acknowledgements}
   \begin{center}
      \footnotesize
      \begin{tabularx}{\textwidth}{CCCC}
         \includegraphics[height = 0.3\paperheight]{figures/nicola_marzari.jpg}     &
         \includegraphics[height = 0.3\paperheight]{figures/nicola_colonna2.png}    &
         \includegraphics[height = 0.3\paperheight]{figures/riccardo_degennaro.jpg} &
         \includegraphics[height = 0.3\paperheight]{figures/yannick_schubert.jpg}     \\
         % \includegraphics[height = 0.2\paperheight]{figures/daniel_cole.jpeg}       &
         % \includegraphics[height = 0.2\paperheight]{figures/mike_payne.jpeg}        &
         % \includegraphics[height = 0.2\paperheight]{figures/david_oregan.jpg}         \\
         Nicola Marzari                                                             &
         Nicola Colonna                                                             &
         Riccardo De~Gennaro                                                        &
         Yannick Schubert                                                             \\
      \end{tabularx}
   \end{center}
   \begin{center}
      \includegraphics[height = 0.15\paperheight]{logos/SNF_logo_standard_print_color_pos_e.eps}
      \hspace{1em}
      \includegraphics[height = 0.15\paperheight]{figures/marvel_trimmed.png}
   \end{center}

   \begin{center}
      \vspace{1em}
      \small
      Want to find out more? Go to \url{koopmans-functionals.org}
      \vspace{1em}

      Next week in Pavia, Italy \emph{Advanced Quantum ESPRESSO school: Hubbard and Koopmans functionals from linear response}. Recordings can be found on the Materials Cloud youtube page
      \vspace{1em}

      Follow \includegraphics[height=\fontcharht\font`\B]{figures/Twitter_Bird.png} \textcolor{twitter_blue}{@ed\_linscott} for updates | Slides available at \includegraphics[height=\fontcharht\font`\B]{logos/github-favicon.png} github/elinscott
   \end{center}

   % \begin{multicols}{2}
   %    \tiny
   %    \printbibliography
   %    \normalsize
   % \end{multicols}
   \vspace{2ex}
   \scriptsize

   \setbeamercolor*{bibliography entry title}{fg=black}
   \setbeamercolor*{bibliography entry author}{fg=black}
   \setbeamercolor*{bibliography entry location}{fg=black}
   \setbeamercolor*{bibliography entry note}{fg=black}

   \vspace{2ex}
   \scriptsize
\end{frame}

\backupbegin
\begin{frame}{}

   \begin{center}
      \huge SPARE SLIDES
   \end{center}

\end{frame}

\begin{frame}{Koopmans functionals: off-diagonal occupancies}
   \begin{block}{Recap from earlier}
      Key idea: construct a functional such that the \emph{variational} orbital energies
      \begin{equation*}
         \varepsilon^\mathsf{Koopmans}_i = \braopket{\varphi_i}{H}{\varphi_i} = \partial E_\mathsf{Koopmans}/\partial f_i
      \end{equation*}
      are...
      \begin{itemize}
         \item independent of the corresponding occupancies $f_i$
         \item equal to the corresponding total energy difference $E_i(N-1) - E(N)$
      \end{itemize}
   \end{block}

   zero band gap $\rightarrow$ occupancy matrix for variational orbitals is off-diagonal
\end{frame}


% \begin{frame}{References}
%    \setbeamercolor*{bibliography entry title}{fg=black}
%    \setbeamercolor*{bibliography entry author}{fg=black}
%    \setbeamercolor*{bibliography entry location}{fg=black}
%    \setbeamercolor*{bibliography entry note}{fg=black}
%    \printbibliography
%    % For further reading on Koopmans functionals, see \cite{Dabo2010,Borghi2014,Nguyen2018,Colonna2018,Colonna2019,DeGennaro2022,Colonna2022}
% 
% \end{frame}

\begin{frame}{Koopmans functionals: results for solids}
   \vspace{-0.5em}
   \begin{figure}[t]
      \centering
      \begin{subfigure}{0.45\textwidth}
         \includegraphics[width=\columnwidth]{figures/Si_kipz_bands.png}
         \caption{Si, KIPZ}
      \end{subfigure}
      \begin{subfigure}{0.45\textwidth}
         \includegraphics[width=\columnwidth]{figures/GaAs_ki_bands.png}
         \caption{GaAs, KI}
      \end{subfigure}
      % \begin{subfigure}{\textwidth}
      %    \begin{tabularx}{\columnwidth}{C C C C C C C}
      %       ZnO                                  & LDA  & HSE  & GW$_0$ & scG$\tilde{\rm W}$ & KI   & exp.      \\
      %       \hline
      %       $E_\mathrm{gap}$ (eV)                & 0.79 & 2.79 & 3.0    & 3.2                & 3.62 & 3.60      \\
      %       $\langle \varepsilon_d \rangle$ (eV) & -5.1 & -6.1 & -6.4   & -6.7               & -6.9 & -7.5/-8.0 \\
      %    \end{tabularx}
      % \end{subfigure}
   \end{figure}
   \begin{center}
      \footnotesize
      \begin{tabular}{l l S[table-format = 2.2] S[table-format = 2.2] >{\color{seaborn_red}\bfseries}S[table-format = 2.2] >{\color{seaborn_red}\bfseries}S[table-format = 2.2] >{\color{seaborn_red}\bfseries}S[table-format = 2.2] S[table-format = 2.2]}
                               &                                  & {PBE} & {QSG$\tilde{\mathsf{W}}$} & {KI} & {pKIPZ} & {\bf KIPZ} & {exp} \\
         \midrule
         \midrule
         {Si}                  & $E_\mathrm{gap}$                 & 0.55  & 1.24                      & 1.18 & 1.17    & 1.19       & 1.17  \\
         \midrule
         \multirow{2}{*}{GaAs} & $E_\mathrm{gap}$                 & 0.50  & 1.61                      & 1.53 & 1.49    & 1.50       & 1.52  \\
                               & $\langle \varepsilon_d \rangle $ & 14.9  & 17.6                      & 16.9 &         & 17.7       & 18.9
      \end{tabular}
   \end{center}
   \blfootcite{DeGennaro2022}
\end{frame}

\begin{frame}{Accelerating improvements: easier Wannierization}
   Separation of target manifolds via parallel transport to obtain separate occupied and empty manifolds

      \begin{figure}
         \onslide<2->{
         \includegraphics[width=0.5\columnwidth]{figures/target_manifolds_fig1b.png}
         }
      \end{figure}

   \blfootcite{Qiao2023a}
\end{frame}


\begin{frame}{Accelerating improvements: screening via ML}
   \begin{center}
      \begin{tikzpicture}
         \node[inner sep=0pt] (water box) at (0,0)
         {
            \includegraphics[width=0.25\textwidth]{figures/orbital.emp.00191_cropped.png}
         };
         \node[below=0cm of water box] (density) {$\rho_i(\mathbf{r})$};
         \node[right=0.2\textwidth of water box] (power spectrum) {
            $
               \begin{bmatrix}
                  x_{0} \\
                  x_{1} \\
                  x_{2} \\
                  \vdots
               \end{bmatrix}
            $
         };
         \path[line] (water box) -- node [midway, above, align=center] (decomposition) {power spectrum \\ decomposition} (power spectrum);
         \node[right=0.2\textwidth of power spectrum] (screening parameter) {$\alpha_i$};
         \path[line] (power spectrum) -- node [midway, above, align=center] (model) {ML model} (screening parameter);
      \end{tikzpicture}
   \end{center}

   \vspace{-4em}

   \blfootcite{Schubert2022}

   \begin{align*}
      c^i_{nlm,k} & =\int d\textbf{r} g_{nl}(r)Y_{lm}(\theta,\varphi)\rho^i(\textbf{r}-\textbf{R}^i)                        \\
      p^i_{n_1n_2l,k_1k_2}         & =\pi \sqrt{\frac{8}{2l+1}}\sum\limits_m {c_{n_1lm,k_1}^{i *}}c_{n_2lm,k_2}^i \label{eq: power spectrum}
   \end{align*}

   % $g_{nl}$ = orthonormalised radial Gaussian basis functions

   % $Y_{lm}$ = spherical harmonics

\end{frame}

\begin{frame}{koopmans: the input file}
   \begin{minipage}[t]{0.475\columnwidth}
      \inputminted[fontsize=\tiny,breaklines,lastline=20]{json}{scripts/si.json}
   \end{minipage}
   \hspace{0.025\textwidth}
   \begin{minipage}[t]{0.475\columnwidth}
      \inputminted[fontsize=\tiny,breaklines,firstline=21]{json}{scripts/si.json}
   \end{minipage}
\end{frame}

% \begin{frame}{koopmans: the output file}
%    \vspace{-2ex}
%    \only<1>{
%       \inputminted[fontsize=\scriptsize,breaklines]{text}{scripts/si_ki_full.out}
%    }
%    \only<2>{
%       \inputminted[fontsize=\scriptsize,breaklines,firstline=25]{text}{scripts/si_ki_full.out}
%    }
%    \only<3>{
%       \inputminted[fontsize=\scriptsize,breaklines,firstline=50]{text}{scripts/si_ki_full.out}
%    }
%    \only<4>{
%       \inputminted[fontsize=\scriptsize,breaklines,firstline=75]{text}{scripts/si_ki_full.out}
%    }
% \end{frame}

\begin{frame}{koopmans is scriptable}
   \vspace{-2ex}
   \inputminted[fontsize=\scriptsize,breaklines]{python}{scripts/si.py}
\end{frame}



\backupend
\end{document}
